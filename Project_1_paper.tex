\documentclass[american,a4paper,12pt]{article}
\usepackage[T1]{fontenc} %for å bruke æøå
\usepackage[utf8]{inputenc}
\usepackage{graphicx} %for å inkludere grafikk
\usepackage{verbatim} %for å inkludere filer med tegn LaTeX ikke liker
\usepackage{mathpazo}

\usepackage{amsmath}
\usepackage{caption}
\usepackage{multicol}
\usepackage{siunitx}
\usepackage{float}
\usepackage{subcaption}


\renewcommand{\vec}[1]{\mathbf{#1}} %ny definisjon av \vec så det blir bold face i stedet for vector-pil.


\captionsetup[table]{skip=10pt}
\bibliographystyle{plain}

\title{Project 1: Computational Physics - FYS3150}
\author{Fredrik Hoftun \& Mikkel Metzsch Jensen}
\date{September 09, 2020}
\begin{document}
\maketitle

\section{Introduction}
  In this project we will investegate different approaches to solve the one-dimensional Poisson equation with Dirichlet boundary conditions given as follows:
  \begin{align*}
    u''(x) = f(x), \quad x \in (0,1), \quad u(0) = u(1) = 0
  \end{align*}
  We will rewrite this as a set of linear equations, and solve it by a number of different computational approaches on either gaussian elimination or LU decomposition. We will solve the equation above with the function:
  \begin{align*}
    f(x) = 100e^{-10x}
  \end{align*}
  Where the analytical solution then is given as:
  \begin{align*}
    u(x) = 1 - (1 - e^{-10})x - e^{-10x}
  \end{align*}
  We will use the analytical solution to evaluate the precision of the numerical solutions for different steplength between the discretized gridpoints $x_i$.
\section{Method}
  \subsection{Rewritting the equation as a set of linear equations}
  In order to solve the Poisson equation numerically we discretize u as $v_i$ with grid points $x_i = ih$ in the interval $x \in [x_0 = 0,x_1 = 1]$. We then have the step length $h = 1/(n + 1)$. We use the following second derivative approximation
  \begin{align*}
    -u''(x_i) \approx -\frac{v_{i+1} + 2v_i - v_{i+1}}{h^2} =  f(x_i)
  \end{align*}
  $\Longleftrightarrow$
  \begin{align*}
    -v_{i-1} + 2v_i - v_{i+1} = h^2f(x_i)
  \end{align*}
  We define the colum vector $\vec{v} = [v_1, v_2, \hdots, v_{n+1}]$ and try to setup the equation for every step i. As we do this we see a pattern appearing
  \begin{align*}
        \begin{bmatrix}
          2 & -1 & 0 & \cdots & 0
        \end{bmatrix}
        \begin{bmatrix}
          v_0 \\
          \vdots \\
          v_{n+1}
        \end{bmatrix}
  = h^2f(x_0)
  \end{align*}
  \begin{align*}
        \begin{bmatrix}
          2 & -1 & 0 & \cdots & \cdots & 0 \\
          -1 & 2 & -1 & 0 & \cdots & \cdots
        \end{bmatrix}
        \begin{bmatrix}
          v_0 \\
          \vdots \\
          v_{n+1}
        \end{bmatrix}
  = h^2
        \begin{bmatrix}
          f(x_0) \\
          f(x_1)
        \end{bmatrix}
  \end{align*}
  \begin{align*}
    \vdots
  \end{align*}
  \begin{align*}
        \begin{bmatrix}
          2 & -1 & 0 & \cdots & \cdots & 0 \\
          -1 & 2 & -1 & 0 & \cdots & \cdots \\
          0 & -1 & 2 & -1 & 0 & \cdots \\
           & \cdots & \cdots & \cdots & \cdots & \cdots \\
          0 & \cdots & & -1 & 2 & -1 \\
          0 & \cdots & & 0 & -1 & 2
        \end{bmatrix}
        \begin{bmatrix}
          v_0 \\
          \vdots \\
          v_{n+1}
        \end{bmatrix}
  = h^2
        \begin{bmatrix}
          f(x_0) \\
          f(x_1) \\
          \vdots \\
          f_{n+1}
        \end{bmatrix}
  \end{align*}
  From this we see that we can write the problem as a linear set of equation:
  \begin{align*}
    \vec{A}\vec{v} = \vec{\tilde{g}}
  \end{align*}
  With the following definitions:
  \begin{align*}
    \vec{A} =
    \begin{bmatrix}
      2 & -1 & 0 & \cdots & \cdots & 0 \\
      -1 & 2 & -1 & 0 & \cdots & \cdots \\
      0 & -1 & 2 & -1 & 0 & \cdots \\
       & \cdots & \cdots & \cdots & \cdots & \cdots \\
      0 & \cdots & & -1 & 2 & -1 \\
      0 & \cdots & & 0 & -1 & 2
    \end{bmatrix}
    \quad, \vec{v} =
    \begin{bmatrix}
      v_0 \\
      v_1 \\
      \vdots \\
      v_{n+1}
    \end{bmatrix}
    \quad, \vec{\tilde{g}} = h^2
    \begin{bmatrix}
      f(x_0) \\
      f(x_1) \\
      \vdots \\
      f_{n+1}
    \end{bmatrix}
  \end{align*}
  In this project we use $f(x) = 100e^{-10x}$. The solution for the Poisson equation in this case is given to be $u(x) = 1 - (1 - e^{-10})x - e^{-10x}$. We can ensure that this is true by inserting it and checking that the equarion holds remains true. First find the double derivative of u(x):
  \begin{align*}
    u'(x) = -(1 - e^{-10}) + 10e^{-10x}, \quad u''(x) = -100e^{-10x}
  \end{align*}
  We now see that the solution satisfy the Poisson equation:
  \begin{align*}
    -u''(x) = 100e^{-10x} = f(x)
  \end{align*}
  \subsection{General solution using Gausian elimination}
    Forward / Backward sub
    FLOPS
  \subsection{Simplified problem specific solution}
    FLOPS
  \subsection{LU decomposition}
    FLOPS
  \subsection{Comparing precision and error}


\section{Implementation?}
\section{Results}
\section{Concluding remarks}








\section{Part a}








\end{document}
