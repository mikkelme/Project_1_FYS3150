\documentclass[american,a4paper,12pt]{article}
\usepackage[T1]{fontenc} %for å bruke æøå
\usepackage[utf8]{inputenc}
\usepackage{graphicx} %for å inkludere grafikk
\usepackage{verbatim} %for å inkludere filer med tegn LaTeX ikke liker
\usepackage{mathpazo}

\usepackage{amsmath}
\usepackage{caption}
\usepackage{multicol}
\usepackage{siunitx}
\usepackage{float}
\usepackage{subcaption}


\renewcommand{\vec}[1]{\mathbf{#1}} %ny definisjon av \vec og \hat så det blir penere


\captionsetup[table]{skip=10pt}
\bibliographystyle{plain}

\title{Project 1: Computational Physics - FYS3150}
\author{Mikkel Metzsch Jensen and Others}
\date{September 09, 2020}
\begin{document}
\maketitle

\section{Part a}
The solution can be shown be doing the following rewritting of the Poisson equation:
\begin{align*}
  -u''(x_i) = f(x_i)
\end{align*}
\begin{align*}
  -\frac{v_{i+1} + 2v_i - v_{i+1}}{h^2} = f(x_i)
\end{align*}
\begin{align*}
  -v_{i-1} + 2v_i - v_{i+1} = h^2f(x_i)
\end{align*}
As we try to setup the equation for f($\vec{v}$) for each individual component the matrix A starts to appear.
\begin{align*}
      \begin{bmatrix}
        2 & -1 & 0 & \cdots & 0
      \end{bmatrix}
      \begin{bmatrix}
        v_0 \\
        \vdots \\
        v_{n+1}
      \end{bmatrix}
= h^2f(x_0)
\end{align*}
\begin{align*}
      \begin{bmatrix}
        2 & -1 & 0 & \cdots & \cdots & 0 \\
        -1 & 2 & -1 & 0 & \cdots & \cdots
      \end{bmatrix}
      \begin{bmatrix}
        v_0 \\
        \vdots \\
        v_{n+1}
      \end{bmatrix}
= h^2
      \begin{bmatrix}
        f(x_0) \\
        f(x_1)
      \end{bmatrix}
\end{align*}
\begin{align*}
  \vdots
\end{align*}
\begin{align*}
      \begin{bmatrix}
        2 & -1 & 0 & \cdots & \cdots & 0 \\
        -1 & 2 & -1 & 0 & \cdots & \cdots \\
        0 & -1 & 2 & -1 & 0 & \cdots \\
         & \cdots & \cdots & \cdots & \cdots & \cdots \\
        0 & \cdots & & -1 & 2 & -1 \\
        0 & \cdots & & 0 & -1 & 2
      \end{bmatrix}
      \begin{bmatrix}
        v_0 \\
        \vdots \\
        v_{n+1}
      \end{bmatrix}
= h^2
      \begin{bmatrix}
        f(x_0) \\
        f(x_1) \\
        \vdots \\
        f_{n+1}
      \end{bmatrix}
\end{align*}
By using the definition from the assignment description we arrive at the the expression
\begin{align*}
  \vec{A}\vec{v} = \vec{\tilde{b}}
\end{align*}
We assume that $f(x) = 100e^{-10x}$. The solution is given to be $u(x) = 1 - (1 - e^{-10})x - e^{-10x}$. We can ensure that this is true by inserting it into the Poisson equation. We first find the double derivative of u(x):
\begin{align*}
  u'(x) = -(1 - e^{-10}) + 10e^{-10x}, \quad u''(x) = -100e^{-10x}
\end{align*}
We now see that the solution satisfy the Poisson equation:
\begin{align*}
  -u''(x) = 100e^{-10x} = f(x)
\end{align*}

\section{Part b}









\end{document}
